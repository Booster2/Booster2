\chapter{Transformations}

\section{Introduction}

A 'staged compilation process', similar to that described on Page 8 of \cite{KatsVisser2010}.

\section{Structure}

\begin{itemize}
  \item Parse
  \begin{itemize}
    \item Initialize lookup table
    \item Populate lookup table
  \end{itemize}
  \item Elaborate
  \begin{itemize}
    \item Insert 'this'
     (requires that every term is annotated with the class it is contained in, 
      and requires a function to lookup method and attribute names) 
    \item Generate inputs and outputs
    \begin{itemize}
      \item Infer Types
      \item Deduce Types
    \end{itemize}
    \item Populate lookup table
    \item Inputs and outputs
    \item Qualified invariants
    \item Class-based invariants
    \item Expanded workflows
    \item (Expand Method References)
    \item (Expand inheritance)
  \end{itemize}    
  \item Compile
  \begin{itemize}
    \item 'Program'
    \item Calculate postcondition
    \item 'WP'
  \end{itemize}
  \item Simplify
  \item Translate
\end{itemize}

\subsection{Parse}

In this initial stage of the process, the tree of abstract terms which
comprises the abstract syntax is inserted into a lookup table, for
later reference.  At this point, only simple inferences are made; any
complicated inferences are left until the 'elaborate' part of the
process.

Before the lookup table can be created, it must first be initialised,
which is useful in later steps: when the lookup-table is
pretty-printed after a stage, perhaps for debugging purposes, there is
no danger of an uninitialised part of the table.
 
\subsection{Elaborate}
\subsection{Compile}
\subsection{Simplify}
\subsection{Translate}
  
\section{Implementing in Spoofax}

\subsection{Explaining the Booster transformations}

Spoofax strategies typically take a 'graph re-writing' pattern: the
tree is iteratively recursed, and when a term matching a particular
pattern is found, an action is performed - typically a re-written
version of the term is replaced in the tree.

However, this approach is not ideal for Booster transformations.
Perhaps fundamentally, this is because the model is a graph, not
really a tree.  When iterating the tree, it is important to have a
great-deal of contextual information present - for example, the list
of attributes and their types for each class.  This contextual
information cannot easily be passed as a parameter during the tree
exploration: it \emph{could} be placed as annotations at suitable nodes in
the tree.  The main problem is that often the model may need updating
somewhere other than the node currently visited.  

These concerns may be illustrated by considering the function which
deals with the expansion of the inheritance hierarchy.  When a node is
found which satisfies the property that

The node is an attribute 'a' in a class 'c1'
The class 'c2' is a sub-class of the class 'c1'
The class 'c2' does not currently contain the attribute 'a'

In this case it is necessary to have the relevant contextual
information available during the examination of a node (which
attributes are in which classes (other than the current), and which
classes are sub-classes of others.  Furthermore, it is necessary to be
able to update this information in such a way that it can be used for
the remainder of the tree-traversal, for otherwise further
applications of this rule may be unapplicable, or, worse, repeated for
the same node under the same pattern matching.

Under these constraints, two implementation choices make themselves
clear.  The first is that a mutable 'lookup-table' is required: a
function which may be updated to reflect the latest understanding and
is always available within any context.

The second is that the use of annotations in transformations such as
this is not immediately useful.  Annotations will need re-using and  
 




\section{A Lookup Table}

There's a blurry line between a system and a booster specification.
Since there is only one System in scope at any one time, then the
lookup table acts as a lookup function for 'System'.

These are the fields that are originally stored in the lookup table.

\begin{verbatim}

"Name" -> name

"SetDef" -> (name, [elements])

"Class" -> (name, ([subclasses], [attributes], [methods], 
                                         [constraints], [workflows]))

"Attribute" -> ((cname, aname), (decorations, type, 
                                         (opposite), minmult, maxmult)) 

"Method" -> ((cname, mname), (constraint, guardedCommand, exts))

\end{verbatim}

Where \verb|exts| in \verb|method| is used solely in the elaboration
phase, to keep track of which constraints from subclasses have been
conjoined into the method definition.
