\chapter{The Booster Language}

This section is intended to document the Booster syntax.  As a
reference, the complete abstract syntax is placed in an appendix.

\section{System, comments}
A Booster system is described by a single text file, with the system
name at the top.  The system name is used as a namespace for any
generated artefacts: databases, services etc. and hence should be both
identifying and unique.

For example:

\begin{code}
system ComputingLaboratory

// ... The system definition goes here
\end{code}

There are no restrictions on the type-senstivity of names in
Booster, but convention is to use camel-case, with the initial letter
capitalised for the names of systems, classes and sets, while methods
and attributes use lower-case letters.

Comments can be (and should be) put into the code in the normal way:
using the familiar notation of \verb|//| for a single-line comment,
and \verb|/* ... */| for multi-line comments.

\begin{code}
// This is a single-line comment

/* This is a multi-line 
comment */
\end{code}

Comments may be placed anywhere inside the text file, and it is
recommended that descriptive comments appear before the code that is
being described.


\section{Classes and attributes}
\label{sec:language-classes}

Booster may be described as an object-based language: data,
functionality and constraints are organised in structures known as
classes.  A class may be defined within the context of a system, and
is given a name, to represent the real-world objects that it will be
representing.  Within a class, attributes, methods and invariants can
be defined within separate sections using the keywords as shown in the
example below:

\begin{code}
class C {
  attributes
    /* attributes for the class C are defined here */
  methods
    /* methods for the class C are defined here */
  invariants
    /* invariants for the class C are defined here */
}
\end{code}

For ease of definition, multiple sections of each type may be
defined.  Each section is optional, although an empty class may
trigger a warning in the editor.  This subsection focuses on the
contents of an \verb|attributes| section; later subsections deal with
methods and attributes.

An attribute definition consists in a name, and information about
type, multiplicity and symmetry constraints.  In addition, there may
be additional hints for an interface about how attributes are used.

A simple attribute is defined by giving its name, and its type,
separated by a colon (\verb|:|).  Here is an example of a simple
attribute:

\begin{code}
  firstName : STRING
\end{code}

This defines an attribute whose name is \verb|firstName| and whose
type is the basic type \verb|STRING|.  In Booster it is conventional
to start attribute names in lower case.  

There is no separator between attribute definitions, e.g.
\begin{code}
  firstName : STRING
  lastName : STRING
\end{code}

Multiplicity information can also be defined.  The examples above show
simple, mandatory attributes.  An attribute may be marked as
\emph{optional} by including the type in square brackets (\verb|[]|).
An optional attribute may take a value included in the type, or may be
given the value \verb|null|, a special value whose presence indicate
that the value of the attribute is not defined.

For example, we may define an optional middle name of type
\verb|STRING|:

\begin{code}
  middleName : [STRING]
\end{code}

We may also allow an attribute to take a number of values.  Initially,
we simply allow an \emph{unordered set} of values, but in future versions we
intend to allow ordered \emph{sets}, \emph{bags}, and \emph{sequences}
of values. 

To define a set-valued attribute, we use the keyword \verb|SET|, round
brackets (\verb|()|), and a multiplicity constraint.  This defines
the maximum and minimum number of values that an attribute may take.
For example, to declare that an attribute has between zero and five
values, we might define as follows:

\begin{code}
  middleNames : SET( STRING )[0..5]
\end{code} 

The syntax for multiplicity constraints is taken from UML.  The table
below shows the different forms of syntax, and the interpretations:

\begin{center}
  \begin{tabular}{ | c | c | c | c |}
    \hline
    \textbf{Description} & \textbf{Syntax Example} & \textbf{Min. Multiplicity} & \textbf{Max. Multiplicity} \\ \hline
    Min. and Max. & \verb|[3 .. 5]| & 3 & 5 \\ \hline
    Max. only & \verb|[ .. 5]| & 0 & 5 \\ \hline
    Min. only & \verb|[3 .. ]| & 3 & No Maximum \\ \hline
    Min. only & \verb|[3 .. *]| & 3 & No Maximum \\ \hline
    Fixed & \verb|[3]| & 3 & 3 \\ \hline
    Any & \verb|[*]| & 0 & No Maximum \\ \hline

  \end{tabular}
\end{center}

Attribute types are discussed in more detail in the next subsection.
One attribute type of particular interest is a reference-valued
attribute, which is a key concept in object-oriented programming.  An
attribute may hold values of references to other objects in the
system.  For example, the attribute \verb|supervisors| of class
\verb|Student| may refer to one or more objects of type \verb|Staff|.
In Booster, this would be defined as follows:

\begin{code}
  class Student
    attributes
      ...
      supervisors : SET(Staff) [1 .. *]
      ...
\end{code}

The final constraint we might wish to add to an attribute is that of
\emph{symmetry}.  This is the property that defines two attributes as
\emph{opposites}, we declare this pair of attributes as representing a
\emph{bi-directional association}.  To illustrate, we assume the
attribute \verb|supervisors| defined in class \verb|Student| as above;
we define the opposite attribute \verb|supervisees| in class
\verb|Staff|.  The constraint we wish to capture is that given a
\verb|Student| object \verb|s1|, for every \verb|Staff| object
referenced in its attribute \verb|supervisors|, \verb|s1| is contained
within its \verb|supervisees|.

To define such a bi-directional association, we simply add the
opposite attribute name as part of the type definition.  For example,
the following code fragment implements the paired association
described above:

\begin{code}
  class Student {
    attributes
      ...
      supervisors : SET(Staff . supervisees) [1 .. *]
      ...
  }

  class Staff {
    attributes
      ...
      supervisees : SET(Student . supervisors) [0 .. *]
      ...
  }
\end{code}

This ensures the symmetry property between the attributes
\verb|supervisors| and \verb|supervisees|.

Finally, we may wish to specify interface-specific annotations that
help interfaces present the information.  Eventually, we'd expect this
information to end up in a separate file.

Decorations are specified in the definition of an attribute,
immediately after the attribute name.  At the moment the only
annotation we have is the 'identity' annotation, which specifies that
we'd like to identify objects in the database using these attributes.
Note that this does not imply any uniqueness constraints.  For
example, we might specify that objects of type \verb|Person| can be
identified by the components of their name.  E.g.:

\begin{code}
class Person {
  attributes
    firstName (ID) : STRING
    lastName (ID) : STRING
    dateOfBirth : [DATETIME]
}
\end{code}

\section{Types and user-defined enumerations}

Booster has four built-in primitive types: \verb|STRING|, \verb|INT|,
\verb|BOOLEAN|, and \verb|DATETIME|.  Section~\ref{sec:addNewType}
describes how to add additional primitive types to the language.  
Class references may also be used as types---as illustrated in the
symmetry description above.

It is often the case that enumerations are required to capture values
of a particular attribute.  In Booster, these enumerations can be
specified using the \verb|set| notation.  A \verb|set| may be defined
in a system for use anywhere within it, and is defined outside of any
class or workflow definitions.

The following example show the definition and use of an enumerated
value:

\begin{code}
system Calendar

set Weekday { Monday, Tuesday, Wednesday, Thursday, Friday}

class Appointment {
  attributes
    ...
    dayOfWeek : WeekDay
    ...
}
\end{code}

The elements of the set may be used as values in the definition of
methods or invariants---see the next section for details.  It is
customary to use upper-case camel case for set names and values.

In the future, it is intended that sets be more flexible: that they
may be extended at runtime.  Values defined in the model may be used
in constraints within the model, but additional values may be added to
the running system.

\section{Methods}

All modifications to the data in a Booster-generated system must be
performed through method calls: where it can be guaranteed that all
business rules and invariants are maintained.  Methods are defined in
the context of classes, and consist of a name and a constraint upon
values in the before- and after-states of the method call.  Methods
should be contained within a \verb|methods| section, inside the class
definition.

The simplest method is the one which is always available, and always
succeeds, defined here within a class \verb|C|, and given the name \verb|m1|:

\begin{code}
class C {
  ...
  methods
    m1 { true }
}    
\end{code}

\subsection{Constraints}
Within the curly brackets is a simple logical constraint.  This may be
the basic constraints \verb|true| and \verb|false| (although typically
these do not appear in real-life systems), or may be the conbination
of other constraints using the familiar conjunction (\verb|&|),
disjunction(\verb|or|), implication (\verb|=>|), or negation
(\verb|not|).  The relational composition operator \verb|;| is also
available for more complex method specifications.  The table below
shows the complete set of logical operators:

\begin{center}
  \begin{tabular} { | c | c | } 
  \hline
  \textbf{Description} & \textbf{Booster Syntax} \\ \hline
  True & \verb|true| \\ \hline
  False & \verb|false| \\ \hline
  Negation & \verb|not| \\ \hline
  Conjunction & \verb|&| \\ \hline
  Disjunction & \verb|or| \\ \hline
  Implication & \verb|=>| \\ \hline
  Composition & \verb|;| \\ \hline
  \end{tabular}
\end{center}
The simplest form of constraint is a comparison between two
expressions.  There are a number of comparison operators: equality
(\verb|=|), less than (\verb|<|) and greater than (\verb|>|) being the
most frequently used.  A complete list of comparison operators is
given in the table below:

\begin{center}
  \begin{tabular} { | c | c | } 
  \hline
  \textbf{Description} & \textbf{Booster Syntax} \\ \hline
  Equal & \verb|=| \\ \hline
  Not equal & \verb|/=| \\ \hline
  Set membership & \verb|:| \\ \hline
  Set non-membership & \verb|/:| \\ \hline
  Less than & \verb|<| \\ \hline
  Less than or equal & \verb|<=| \\ \hline
  Greater than & \verb|>| \\ \hline
  Greater than or equal & \verb|>=| \\ \hline
  Strict subset & \verb|<:| \\ \hline
  Subset & \verb|<<:| \\ \hline
  Strict superset & \verb|:>| \\ \hline
  Superset & \verb|:>>| \\ \hline
  \end{tabular}
\end{center}

\subsection{Expressions}

Expressions make up the two sides of any comparison.  The simplest
form of expressions are defined as \emph{value expressions},
corresponding to those expressions whose value may be simply
evaluated.  For example, the integer \verb|1| or the string
\verb|"Hello, World"| are both examples of value expressions.

Expressions may also refer to the values of attributes in the model:
for example we may refer to \verb|o.a| to
refer to the values stored in the attribute \verb|a| for the object
\verb|o|.  Path expressions may be created using the dot (\verb|.|)
notation: if \verb|o.a| is also an object reference, we might refer to
\verb|o.a.b|.  The familiar keyword \verb|this| is used to refer to
the current object.

Methods may also make use of input values (typically identifying
parameters to a method), and output values (typically specifying
objects created during the execution of a method).  Inputs and
outputs are denoted by the syntax \verb|?| and \verb|!| respectively.
For example, \verb|a?| represents an input named \verb|a|.  Input and
ouptut parameters do not need to be declared; instead the compiler
finds all inputs and outputs for a given method and deduces their
types from their usage.  This is explained in more detail in
Chapter~\ref{sec:typeDeduction}.

Methods may constrain values in both the before- and after-states.
For example, a method \verb|m1| may constrain the attribute \verb|a|
to be less than \verb|7| before the method is called and to be
incremented by 1 after the method is called.  In Booster, we denote
values in the post-state with the decoration ``\verb|'|'', which will
be familiar to users of the Z notation.  The constraint described
above might be written as:

\begin{code}
  m1 { a < 7 & a' = a + 1 }
\end{code}

This states that the value of \verb|a| before the
method is called is less than 7, and the the value of \verb|a| after
the method is called is equal of the value of \verb|a| before the
method is called, incremented by one.

There is no need to order predicates within a method constraint, but
it is conventional to place those applying just to the precondition
first, for ease of reading.

The complete set of possible value expressions are given in the table
below:

\begin{center}
  \begin{tabular} { | c | c | } 
  \hline
  \textbf{Description} & \textbf{Booster Syntax Example} \\ \hline
  Primitive value, such as a string or integer & \verb|13|,
  \verb|"Hello World!"| \\ \hline
  Value taken from a defined set & \verb|Tuesday| \\ \hline
  Null value: an undefined optional attribute & \verb|null| \\ \hline
  Type name: the set of all instances of that type & \verb|INT|, \verb|Student|, \verb|Weekday| \\ \hline
  Method input or output & \verb|a?|, \verb|b!| \\ \hline
  Path expression & \verb|this . a . b|, \verb|a? . c| \\ \hline
  Set of expressions & \verb|{ 1 , 3 , 5 , 7 }|, \verb|{}| \\ \hline
  \end{tabular}
\end{center}

Expressions may be combined through the use of operators.  The list of
operators is constantly changing, and should be adapted to fit the
collection of primitive types in use.  This is explained in more
detail in Section~\ref{sec:addNewExpressionOp}. A default set of
common operators are shown in the table below:
\begin{center}
  \begin{tabular} { | c | c | } 
  \hline
  \textbf{Description} & \textbf{Booster Syntax Example} \\ \hline
  Head element of a sequence, or a string & \verb|head(e)| \\ \hline
  Last element of a sequence, or a string & \verb|tail(e)| \\ \hline
  Size of a set, sequence, or string & \verb|card(e)| \\ \hline
  Negation of a numeric value & \verb| -e | \\ \hline
  Addition of two values & \verb| e1 + e2 | \\ \hline
  Difference between two values & \verb| e1 - e2 | \\ \hline 
  Muliplication of two numeric values & \verb| e1 * e2 | \\ \hline 
  Division of two numeric values & \verb| e1 / e2 | \\ \hline 
  Maximum of two values & \verb| e1 max e2 | \\ \hline 
  Minimum of two values & \verb| e1 min e2 | \\ \hline 
  Intersection of two set values & \verb| e1 /\ e2 | \\ \hline 
  Union of two set values & \verb| e1 \/ e2 | \\ \hline 
  Concatenation of two sequence or string values & \verb| e1 ++ e2 | \\ \hline 
 \end{tabular}
\end{center}

The full syntax of expressions can be found in the appendices.


\subsection{Method References}

In order to simplify and rationalize the definition of method
constraints, Booster allows the re-use of one method constraint inside
the definition of another method.  In this simplest case, this just
requires using the method name in place of a predicate.  For example,
we might define the method \verb|increment| as follows:

\begin{code}
  increment { a' = a + 1 }
\end{code}

and then re-use it in the subsequent method \verb|maybeIncrement|:

\begin{code}
  maybeIncrement { a < 5 => increment() }
\end{code}

which performs the increment method if the value of attribute \verb|a|
has a value less than five before the method is called.

The rounded brackets indicate that we may also pass parameters to this
method definition: essentially determining the values of any inputs to
the method being referred to.  For example, given the method
specifications:

\begin{code}
  increment { a' = a + i? }

  maybeIncrement { a < 5 => increment(i? = 5) }
\end{code}

the method \verb|maybeIncrement| would be equivalent to the following
definition:

\begin{code}
  maybeIncrement { a < 5 => a' = a + 5 }
\end{code}

The parameter list is a comma-separated list of input names, and the
expressions that are to be substituted, using the \verb|=| sign to
assert their equivalence.

We may also refer to a method of another object, by substituting a
value for \verb|this|.  By convention, the notation is slightly
different here - we place a path expression in front of the method
name.  For example, we might define:

\begin{code}
  maybeIncrementOther { a < 5 => obj? . increment(i? = 5) }
\end{code}

which includes the specification of \verb|increment| from the class of which
\verb|obj?| is a member.  Note that this is equivalent to the more verbose:

\begin{code}
  maybeIncrementOther { a < 5 => increment(this = obj?, i? = 5) }
\end{code}

which may prove confusing where a method named \verb|increment| may be
defined on multiple classes.

\section{Invariants}

Invariants can be implicit, such as the constraints upon type,
multiplicity and symmetry, as defined above.  Alternatively, they can
be user-defined, in one of the three forms: \emph{static},
\emph{dynamic}, and \emph{derived attributes}.  The first of these is
the most generic, allowing arbitrary constraints to be placed on
inter-connected classes.  Dynamic invariants are a special form of
static invariants, whose structure allows the heuristics to enhance
the behaviour of methods, rather than constraining their
availability.  The effect of these is discussed fully in
Section~\ref{sec:dynamicInvs}.  Finally, derived attributes are a
particular class of dynamic invariant whose use is common enough to
warrant a special syntax.



\subsection{Static Constraints}
An invariant provides a constraint upon the class in which it is
defined.  Such a constraint is defined in the \verb|invariants|
section of a class, and again, multiple sections may be defined within a
single class definition.  Invariants are defined using the same constraint
language used for methods, described above.

For example, we may wish to express a constraint that the radius of a
circle is always non-negative.  Such an invariant could be defined as
follows:

\begin{code}
class Circle {
  attributes
    radius : INT
  invariants
    radius >= 0
}
\end{code}

Note that within an invariants clause, the context of any path is
always \verb|this|, and there is an implicit quantification which
states that this constraint holds for all objects \verb|this| of the
class.

Multiple invariants may be defined using conjunction, but for
convenience, the Booster language allows a list of invariants which
are implicitly conjoined.  For example, consider the
further-constrained definition:

\begin{code}
class Circle {
  attributes
    radius : INT
  invariants
    radius >= 0
    radius < 100
}
\end{code}

Which describes a class of \verb|Circle| whose objects must have an
attribute whose radius is both non-negative and less than \verb|100|.

\subsection{Dynamic Invariants}

TODO!

\begin{itemize}
\item Dynamic invariants may refer to post-state attributes.
\item There's an implication, where the antecedent gives a condition,
  and the consequent specifies a further constraint which applies.
  Typically, this consequent is intended as a postcondition which can
  be translated into an extra guarded command.
\end{itemize}

There's an example where if we put something into a set \verb|S|, then
we must take it out of \verb|T|:
\begin{code}
class A
  dynamic invariants
    S' \/ T' /= {} => T' = T' \ S'
\end{code}

Another example might be something inspired by aspect-oriented
programming.  Changing the 'last updated' attribute is fairly easy:

\begin{code}
class A {
  attributes
    x : INT
    y : INT
    lastUpdated : DATETIME
  dynamic invariants
    x /= x' or y /= y' => lastUpdated' = CurrentDateTime
}
\end{code}

  
\subsection{Derived attributes}

TODO!

A derived attribute is a special form of dynamic invariant.  When an
attribute's value may always be calculated from the value of other
attributes, we could write this as a dynamic invariant with a
\verb|true| antecedent.  For example, to assert that the value of the
attribute \verb|noOfChildren| is always equal to the size of the set
of children denoted by the attribute \verb|children|, we could state
a dynamic invariant:

\begin{code}
invariants
  ...
  true => noOfChildren' = card(children')
\end{code}

In the case where the attribute \verb|noOfChildren| is not updated in
any other way, Booster allows the constraint to be declared as part of
the attribute declaration.  For example, the constraint above may be
specified as:

\begin{code}
class Parent {
  attributes
    children : SET(Child)[*]
    noOfChildren = card(children)
}
\end{code}

Note that for these derived attributes, the type of the attribute is
deduced automtically.

\section{Extension}

Extension in Booster allows the modeller to capture the 'is a'
relationship between classes.  Put simply, if a class \verb|B| extends
a class \verb|A|, then any attributes, methods and constraints that
describe \verb|A| also describe \verb|B|.  Class \verb|B| may also
have additional attributes, methods or constraints that describe the
differences between it and class \verb|A|.

For example, consider the following intuitive example:

\begin{code}
class Rectangle {
  attributes
    width : INT
    height : INT
}

class Square extends Rectangle {
  invariants
    width = height
}
\end{code}

In practical terms, this would be equivalent to the following
definition for \verb|Square|:

\begin{code}
class Square {
  attributes
    width : INT
    height : INT
  invariants
    width = height
}
\end{code}

This notion of extension is a good way of structuring models to
maximise re-use.  However, it's also beneficial because any method
which was applicable to the class \verb|Rectangle| is equally
applicable to the type \verb|Square|, for example in the following
method:

\begin{code}
addShapeToCollection { s? : myShapes' }
\end{code}

the input parameter \verb|s?| may be an object of either type
\verb|Rectangle|, or type \verb|Square|.

Methods in a sub-class may further constrain the method defined in a
super-class.  For example:

\begin{code}
class A {
  ...
  methods
    m1 { x' = 5 }
}
class B extends A {

  methods
    m1 { x < 5 }
}
\end{code}

In this example, the method \verb|m1| in class \verb|B| receives the
following constraint:

\begin{code}
    m1 { x < 5 & x' = 5 }
\end{code}

Note that methods and invariants cannot be weakened in a sub-class:
this enforces the 'is-a' relationship and enables a compositional
approach to factorizing code.

In Booster it makes sense to allow multiple inheritance.  In this
case, attributes, methods and invariants from all super-classes are
propagated to the sub-class.  For example, consider the following
model:

\begin{code}
class Staff {
  attributes
    supervisees : SET(Student . supervisor)[*]
}

class Student {
  attributes
    supervisor : Staff . supervisees

}

class TA extends Student, Staff {
  invariants
    supervisor /= this
}
\end{code}

Where an attribute is inherited multiple times (imagine classes
\verb|Staff| and \verb|Student| both extend a common class
\verb|Person|), then each attribute appears only once.



\section{Workflows}


