\chapter{The Booster Language}


This section is intended to document the Booster syntax.  As a
reference, the complete abstract syntax is placed in an appendix.

\section{System, comments}
A booster system is described by a single text file, with the system
name at the top.  The system name is used as a namespace for any
generated artefacts: databases, services etc. and hence should be both
identifying and unique.

For example:

\begin{code}
system ComputingLaboratory

// ... The system definition goes here
\end{code}

There are no restrictions on the type-senstivity of names in
Booster, but convention is to use camel-case, with the initial letter
capitalised for the names of systems, classes and sets, while methods
and attributes use lower-case letters.

Comments can be (and should be) put into the code in the normal way:
using the familiar notation of \verb|//| for a single-line comment,
and \verb|/* ... */| for multi-line comments.

\begin{code}
// This is a single-line comment

/* This is a multi-line 
comment */
\end{code}

Comments may be placed anywhere inside the text file, and it is
recommended that descriptive comments appear before the code that is
being described.


\section{Classes and attributes}

Booster may be described as an object-based language: data,
functionality and constraints are organised in structures known as
classes.  A class may be defined within the context of a system, and
is given a name, to represent the real-world objects that it will be
representing.  Within a class, attributes, methods and invariants can
be defined within separate sections using the keywords as shown in the
example below:

\begin{code}
class C {
  attributes
    /* attributes for the class C are defined here */
  methods
    /* methods for the class C are defined here */
  invariants
    /* invariants for the class C are defined here */
}
\end{code}

For ease of definition, multiple sections of each type may be
defined.  Each section is optional, although an empty class may
trigger a warning in the editor.  This subsection focuses on the
contents of an \verb|attributes| section; later subsections deal with
methods and attributes.



\section{Types and user-defined enumerations}

\section{Methods}

\section{Invariants: Static and Dynamic}

\section{Inheritance}