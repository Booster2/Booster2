\chapter{Good Modelling}

\section{Bi-directional associations}

Mandatory-to-mandatory associations are difficult to work with, and
often imply there is something wrong with the model!
Object can't be created, except in pairs; they can't be deleted, except
in pairs, and unless some careful swapping is arranged, links are
difficult to update.  Usually mandatory-to-mandatory associations can
be refactored into one conjoined class, or may be better expressed as
an optional-mandatory relationship.

Using bi-directional associations is usually the right thing to do.
It's very rare that you want to know what's in a set, but you don't
want to know what set a thing is in.  Navigation on a website is so
much better if it works both ways.  There are some extreme examples
though - consider the model:

\begin{code}
class Country {
  attributes
    name : STRING
}

class Person {
  attributes
    bornIn : Country
}
\end{code}

Here, it may be appropriate to ignore the back-link, since it's
unlikely to ever require the set of all people who were born in a
country.  This illustrates cases where the use of a class is really as
an extensible enumeration, perhaps with extra attributes.

If backward-navigability is not required in the interface, it's more
appropriate to disable this at the interface level, rather than
constraining the model.


\section{Inheritance and extension}

Designed to capture the 'is-a' relationship, not a generic mechanism
for extension.  For example, a square is a rectangle, and so the
extension is appropriate:

\begin{code}
class Rectangle {
  attributes
    width : INT
    height : INT
}

class Square extends Rectangle {
  invariants
    width = height
}
\end{code}

Compare this with the following extension of Rectangle, which doesn't
conform to the 'is-a' relationship:

\begin{code}
class Cuboid extends Rectangle {
  attributes
    depth : INT
}
\end{code}

Whilst this makes sense in its current state, imagine a later revision
to include the derived attribute \verb|area| in class
\verb|Rectangle|:

\begin{code}
class Rectangle {
  attributes
    ...
    area = width * height
}
\end{code}

This makes sense for the class \verb|Square| but doesn't make sense
for the class \verb|Cuboid|.
