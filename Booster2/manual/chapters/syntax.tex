\chapter{The Booster Syntax}

The concrete syntax is presented here for reference.  The comments
here are included in the source files.  The syntax is presented in a
depth-first fashion for ease of reading.

\section{Booster.sdf}

We import some definitions from the Abstract Booster Model (which
includes syntax for guarded commands), and from the relational model,
which includes an SQL syntax.

\begin{code}
imports Common AbstractBoosterModel Relational
\end{code}

The start token for any instance is the \verb|system| token.

\begin{code}
context-free start-symbols
  System
\end{code}  	  	

A system consists in the keyword \verb|System|, along with an
identifier, and a set of system components.
\begin{code}
'system' ID ( SystemComponent )* -> System {cons("System")}

System -> BoosterTerm
\end{code}

Note that every construct in the language is a 'Booster term', which
is a hook to allow some concrete syntax transformations.

System components are either set definitions, or class definitions.
\begin{code}
SetDef -> SystemComponent
Class  -> SystemComponent

SystemComponent -> BoosterTerm
\end{code}

A set definition begins with the keyword \verb|set|, and an
identifier.  A non-empty list of enumerated values is given, within
curly brackets (\verb|{}|). 

\begin{code}	 
'set' ID '{' {ID ','}+ '}' -> SetDef {cons("SetDef")}

SetDef -> BoosterTerm
\end{code}

A class is defined with the keyword \verb|class| and an identifier.
There is an optional extension clause, and then a list of class
components, surrounded in curly brackets (\verb|{}|).

\begin{code}
'class' ID Extends? '{' ClassComponents * '}' 
                                  -> Class  {cons("Class")}

Class -> BoosterTerm
\end{code}

An extends clause is the keyword \verb|extends|, and a list of class
names, which must be non-empty.

\begin{code}
'extends' {Extension ','}+ -> Extends {cons("Extend")}

Extends -> BoosterTerm

ID -> Extension {cons("Extension")}
\end{code}

A class component is a section containing attributes, methods,
invariants or workflows.  Each section is prefixed by the appropriate
keyword. 

\begin{code}
'attributes' Attribute*        -> ClassComponents {cons("Attributes")}
'methods' Method*              -> ClassComponents {cons("Methods")}
'invariants' Constraint*       -> ClassComponents {cons("Invariant")}
'workflows' WorkflowComponent* -> ClassComponents {cons("Workflows")}

ClassComponents -> BoosterTerm
\end{code}	


An attribute declaration contains an identifier, an optional list of
decorations, and a type declaration.  The type is separated by a colon
\verb|:|.

\begin{code}
ID Decoration* ':' TypeDecl -> Attribute {cons("Attribute")}

Attribute -> BoosterTerm
\end{code}

At the moment, the only decoration currently supported is that of
identity, which is simply the keword \verb|ID|, surrounded in
brackets.

\begin{code}
'(' 'ID' ')' -> Identity {cons("Identity")}

Identity -> Decoration
\end{code}

A type declaration uses the 'primitive' type declaration in one of
three forms: undecorated, to mean mandatory; in square brackets
(\verb|[]|) to indicate optionality; with the keyword \verb|set|, and
round brackets, to indicate a set-valued attribute.  In the latter
case, a mutiplicity is also required in square brackets (\verb|[]|).

\begin{code}
PrimTypeDecl               -> TypeDecl  
'[' PrimTypeDecl ']'       -> TypeDecl {cons("Optional")} 
'set' '(' PrimTypeDecl ')' '[' Multiplicity ']' 
                           ->  TypeDecl {cons("Set")}

TypeDecl -> BoosterTerm
\end{code}

A primitive type declaration is either one of the basic types; or it
is the name of a class, or set definition; or a class identifier with
an attribute name, signifying the opposite attribute.

\begin{code}
BasicTypeDecl -> PrimTypeDecl {cons("BasicType")}

ID            -> PrimTypeDecl {cons("UniDirectional")}
ID '.' ID     -> PrimTypeDecl {cons("BiDirectional")}

PrimTypeDecl -> BoosterTerm
\end{code}

A basic type is either a string, an integer, a datetime, or a boolean
value:
\begin{code}
'STRING'   -> BasicTypeDecl {cons("String")}
'INT'      -> BasicTypeDecl {cons("Int")}
'DATETIME' -> BasicTypeDecl {cons("DateTime")}
'BOOLEAN'  -> BasicTypeDecl {cons("Boolean")}
\end{code}	 

Mupliticties can come in many forms.  See
section~\ref{sec:language-classes} for more details.  Here the
different formalisms are enumerated explicitly:

\begin{code}
INT '..' INT -> Multiplicity {cons("MultMinAndMax")}
    '..' INT -> Multiplicity {cons("MultJustMax")}
INT '..'     -> Multiplicity {cons("MultJustMin")}
INT '..' '*' -> Multiplicity {cons("MultJustMin")}
    INT      -> Multiplicity {cons("MultSingle")}
    '*'      -> Multiplicity {cons("MultAny")}

Multiplicity -> BoosterTerm
\end{code}


A method is defined with an identifier, and a constraint within curly
brackets (\verb|{}|).  For the purposes of auto-completion, it is
useful to define allow an expression instead of a constraint: this is
marked as deprecated to warn the user.  For the purposes of debugging,
a method body may also be defined as a guarded command.

\begin{code}
ID '{' Constraint '}'     -> Method {cons("Method")}
ID '{' Expression '}'     -> Method {cons("Method"), deprecated}
ID '{' GuardedCommand '}' -> Method {cons("Method")}

Method -> BoosterTerm
\end{code}

In its simplest form, a constraint is either \verb|true|,
\verb|false|, or a relation between expressions: 

\begin{code}
'true'                          -> Constraint {cons("True")}
'false' 			-> Constraint {cons("False")}
Relation 			-> Constraint 
\end{code}

Alternatively, a constraint may be made up of smaller constraints,
using the logical combinators \emph{not}, \emph{and}, \emph{or},
\emph{implies}, or \emph{relational composition}.  A constraint may
also appear in brackets, to denote operator precedence.

\begin{code}
'not' Constraint		-> Constraint { cons("Not")}
Constraint '&' Constraint 	-> Constraint { cons("And"), assoc}
Constraint 'or' Constraint 	-> Constraint { cons("Or"), assoc}
Constraint '=>' Constraint 	-> Constraint { cons("Implies"), assoc}
Constraint ';' Constraint 	-> Constraint { cons("Then"), assoc}
"(" Constraint ")" 		-> Constraint { bracket}
\end{code}

More complex combinators are provided in the quantifiers: existential
and unviersal.  These use the keywords \verb|exists| and \verb|forall|
respectively, and declare a quantified variable, a set expression
denoting the range, and a constraint:
\begin{code}
'exists' ID ":" Expression "@" Constraint -> Constraint {cons("Exists")}
'forall' ID ":" Expression "@" Constraint -> Constraint {cons("Forall")}
\end{code}

The final type of constraint is that of a method reference, which is
defined later.
\begin{code}
MethodReference -> Constraint {cons("MethodRef")}

Constraint -> BoosterTerm 
\end{code}
	 
A relation, or comparison between two expressions, is simply the two
expressions separated by an operator:

\begin{code}
Expression BinRel Expression -> Relation {cons("BinRel")}

Relation -> BoosterTerm
\end{code}  

The list of relational operators are as follows:	 

\begin{code}
'=' 	-> BinRel {cons("Equal")}
'/=' 	-> BinRel {cons("NotEqual")}
':' 	-> BinRel {cons("In")}
'/:' 	-> BinRel {cons("NotIn")}
'<' 	-> BinRel {cons("LessThan")}
'>' 	-> BinRel {cons("GreaterThan")}
'<=' 	-> BinRel {cons("LessThanEquals")}
'>=' 	-> BinRel {cons("GreaterThanEquals")}
'<:' 	-> BinRel {cons("Subset")}
'<<:' 	-> BinRel {cons("SubsetEquals")}
':>' 	-> BinRel {cons("Superset")}
':>>' 	-> BinRel {cons("SupersetEquals")}
\end{code}
    
An expression is either a simple value expression, or is formed of sub
expressions: either through a unary operator or a binary operator -
the lists of these are defined below.  Brackets are also permitted to
declare precedence.

\begin{code}    
ValueExpression 	    -> Expression
UnOp Expression	            -> Expression {cons("UnOp")}
Expression BinOp Expression -> Expression {cons("BinOp"), left}
"(" Expression ")" 	    -> Expression {bracket}

Expression -> BoosterTerm
\end{code}

A value expression may be either a basic value, such as an integer or
string; a type extent, such as the set of all integers; a path
expression; the \emph{null} value; or a (possibly empty) set of
further expressions.  In the final case, the expressions are presented
in a comma-separated list, surrounded by curly brackets (\verb|{}|):

\begin{code}
BasicValue                 -> ValueExpression {cons("BasicValue")}
TypeExtent                 -> ValueExpression {cons("TypeExtent")}
Path		           -> ValueExpression {} 
'null'		           -> ValueExpression {cons("Null"), prefer}
"{" {Expression "," }* "}" -> ValueExpression {cons("SetExtent")}

ValueExpression -> BoosterTerm
\end{code}

A basic value can either be a string, or an integer:
\begin{code}
INT 		-> BasicValue {cons("Integer")}
STRING		-> BasicValue {cons("String")}
\end{code}

A type extent is simply the name of one of the basic types:
\begin{code}
'string'	-> TypeExtent {cons("String"), prefer}
'int'		-> TypeExtent {cons("Int"), prefer}
'datetime'	-> TypeExtent {cons("DateTime"), prefer}
'boolean'	-> TypeExtent {cons("Boolean"), prefer}
\end{code}

A path is either a \emph{path start}, or it is a path followed by a
path component, using the standard 'dot' (\verb|.|) notation.

\begin{code}
PathStart -> Path
Path '.' PathComponent -> Path {cons("Path"), prefer}

Path -> BoosterTerm
\end{code}

A \emph{path start} can be an input, or an output, the keyword
\emph{this} in a primed or unprimed fashion, or an attribute
identifier, which again, may be optionally primed.

\begin{code}
Input         -> PathStart
Output        -> PathStart
This          -> PathStart
ThisPrimed    -> PathStart
ID Primed? -> PathStart {cons("PathStart")}

PathStart     -> BoosterTerm
\end{code}

Inputs are simply an identifier with the \verb|?| decoration; ouptuts
are identifiers with the \verb|!| decoration.

\begin{code}
ID "?" -> Input {cons("Input")}
ID "!" -> Output {cons("Output")}
\end{code}

The keyword \verb|this| is used to indicate the current object.  It
may appear in primed form.

\begin{code}
'this' -> This {cons("This")}
'this' "'" -> ThisPrimed {cons("ThisPrimed")}
\end{code}

The primed decoration is simply the character \verb|'|.

\begin{code}
"'" -> Primed {cons("Primed")}
	
Primed -> BoosterTerm
\end{code}

A path component is simply an attribute name, which may optionally be
primed:

\begin{code}
ID Primed? -> PathComponent {cons("PathComponent")}
\end{code}

For auto-completion purposes, it is handy to understand the unfinished
path: the empty path component.
\begin{code}
  -> PathComponent {cons("PathComponent"), deprecated}

PathComponent -> BoosterTerm
\end{code}

The list of unary expression operators is given below:

\begin{code}
'head' -> UnOp {cons("Head")}
'tail' -> UnOp {cons("Tail")}
'card' -> UnOp {cons("Cardinality")}
'-'    -> UnOp {cons("Negative")}
\end{code}

The list of binary expression operators is given below:

\begin{code}
"+"	-> BinOp {cons("Plus")}
"-" 	-> BinOp {cons("Minus")}
"*" 	-> BinOp {cons("Times")}
"/" 	-> BinOp {cons("Divide")}
"max" 	-> BinOp {cons("Maximum")}
"min" 	-> BinOp {cons("Minimum")}
"/\\" 	-> BinOp {cons("Intersection")}
"\\/" 	-> BinOp {cons("Union")}
"++" 	-> BinOp {cons("Concat")}
\end{code}


A method reference consists in a path, which denotes the name of the
method, and the object on which the method is to be performed.  A list
of substitutions is provided in brackets: an input name and an
expression it is to be replaced with, separated by an equals
(\verb|=|) sign.

\begin{code}
Path '(' { ( ID '?' '=' Expression) "," }* ')' 
                            -> MethodReference {cons("MethodReference")}

MethodReference -> BoosterTerm
\end{code}

A workflow component is either a sequential workflow, or a parallel
workflow.

\begin{code}
SeqWf -> WorkflowComponent
ParWf -> WorkflowComponent
\end{code}

Both sequential and parallel workflows are introduced with the
keywords \verb|seq| and \verb|par| respectively.  An identifier is
provided, followed by a colon (\verb|:|).  Finally an expression
of the appropriate type is given.

\begin{code}
'seq' ID ':' SeqWfExpression -> SeqWf {cons("SeqWf")}
'par' ID ':' ParWfExpression -> ParWf {cons("ParWf")}
\end{code}

A sequential workflow expression is either the keyword \verb|skip|, or
the choice between two guarded actions, or a wait expression:
\begin{code}
'Skip'                                   -> SeqWfExpression {cons("Skip")}

GuardedAction '->' SeqWfExpression
'[]' 
GuardedAction '->' SeqWfExpression       -> SeqWfExpression{cons("Choice")}

'WAIT.' INT '.' INT '->' SeqWfExpression -> SeqWfExpression {cons("Wait")}
\end{code}

A sequential workflow may also be defined as a reference to another
sequential workflow, a guarded sequential workflow, or may be
bracketed to resolve ambiguities.  These definitions are intended to
be syntactic sugar, and should be simplified before any precondition
calculation is attempted.

\begin{code}
ID				   -> SeqWfExpression {cons("WorkflowReference")}  
GuardedAction '->' SeqWfExpression -> SeqWfExpression {cons("Prefix")}
"(" SeqWfExpression ")" 	   -> SeqWfExpression {bracket}
\end{code}

A guarded action is simply a guard expression, followed by a method
reference:

\begin{code}
Guard '&' MethodReference -> GuardedAction {cons("GA")}
\end{code}

And a guard is a constraint preceded by the keyword \verb|Normal| or
\verb|Delayed|.  A guard may also be bracketed to resolve ambiguities.

\begin{code}
'Normal' Constraint  -> Guard {cons("Normal")}
'Delayed' Constraint -> Guard {cons("Delayed")}
"(" Guard ")"        -> Guard {bracket}
\end{code}  

Parallel workflows may be a reference to another workflow, or the
parallel combination of another workflow and a parallel workflow
expression.

\begin{code}    
ID			 -> ParWfExpression {cons("Single")}
ID '|||' ParWfExpression -> ParWfExpression {cons("Multiple")}
\end{code}

Some priorities are set to help the parser disamiguate complex constraints:

\begin{code}
context-free priorities
  'not' Constraint		-> Constraint 
  > Constraint '&' Constraint 	-> Constraint 
  > Constraint 'or' Constraint 	-> Constraint
  > Constraint '=>' Constraint 	-> Constraint 
  > Constraint ';' Constraint 	-> Constraint 
\end{code}


\section{AbstractBoosterModel.sdf}
\section{Relational.sdf}
