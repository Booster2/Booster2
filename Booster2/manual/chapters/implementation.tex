\chapter{Implementation}

\section{Introduction}

In this section we discuss the specifics of the Spoofax implementation

\section{Compilation}

The spoofax transformations can be executed in one of two ways: either
as a compiled, java program, or as an interpreted CTree.  Brief details
can be found here: \url{http://strategoxt.org/Spoofax/FAQ}.  By
default, new Spoofax projects use the CTree implementation; we prefer
the java implementation as it is more portable: in particular, the
test framework uses a compiled jar file.

To change between CTree and Java implementations, you need to change
the project in two places: the Ant script that compiles the
transformations, and the code which tells the editor how to apply
transformations to a \verb|.boo2| file.

The first change is in the file \verb|build.main.xml|, and you need to
change line 43, which determines the main targets for the ant build.
To use a Java implementation, the line must be as follows:
\begin{code}
<!-- Main target -->
<target name="all" depends="meta-syntax, spoofaximp.default.jar"/>
\end{code}

Whereas if you require a CTree interpreter, this line can be
re-written:

\begin{code}
<!-- Main target -->
<target name="all" depends="meta-syntax, spoofaximp.default.ctree"/>
\end{code}

Note that of course you could instruct the builder to do both, but
this makes the build take twice as long, which is inconvenient when
developing and testing the transformations.

The second step is to change which transformation engine is provided
by the editor for files with an extension of \verb|.boo2|.  To do
this, you need to edit the file \verb|editor\Booster2-Builders.esv|
and replace the line:

\begin{code}
provider : include/booster2.ctree  
\end{code}

with the lines:

\begin{code}
provider : include/booster2-java.jar                                                                                      
provider : include/booster2.jar  
\end{code}

Or vice-versa, if you're planning on using the CTree interpreter.
Remember that the project will need re-building after these two
changes have been made.

To execute the transformations on the command line, you will need the
java implementation.  Ensure that your files
\verb|include/booster2.jar| and \verb|include/booster2-java.jar| are
up to date.  Then from the top-level directory, you can execute the
command (all as one line):

\begin{code}
java -cp 
include/booster2.jar:include/booster2-java.jar:utils/strategoxt.jar
run trans.java-elaborate-booster -i test.boo2
\end{code}

where \verb|run| is the java class that gets executed (defined in
\verb|strategoxt.jar|), \verb|java-elaborate-booster| is the name of
the transformation that you'd like to apply, and \verb|test.boo2| is
the input source file.  The transformation that gets applied must be
appropriately wrapped in a 'main' function as described in
\url{http://strategoxt.org/Spoofax/CommandLine} which allows input to
taken from the command line, and calls the top-level parser (to
generate the initial ATree) manually. 

\section{Testing}

The tests are defined as \emph{regression tests} - i.e. they are in
place to ensure that after any changes to the transformations are
made, the existing functionality continues to work as expected.
Currently these tests take the form of a booster source file, and a
set of expected results: additional source files which should match
the expected result of any of the key transformation stages.  For
example, for a given input file \verb|test.boo2|, there would be
suitable result files \verb|test.parsed.boo2|,
\verb|test.elaborated.boo2|, etc.

The ANT files which runs the tests can be found in
\verb|test/regression/runTests.xml|.  This process first empties the
results of any previous testing (which are kept so as to be available for
inspection after testing), and then runs the test again.  For each
source file it finds in the \verb|source| directory, it runs each
transformation and moves the result into the output directory.  It
uses the host machine's \verb|diff| function to compare the output
with the expected result: using options to ignore white space and
empty lines where applicable, to allow for small changes in the output
format.  Based on the result of this \verb|diff| command, it moves the
file into the appropriate \verb|success| or \verb|failure| directory
for later inspection.  

Finally, the process counts the number of files in the \verb|success|
and \verb|failure| directories to provide a test report.

